\documentclass{article}
\usepackage[utf8]{inputenc}
\usepackage{amsmath}
\begin{document}
 \section{Diagrama}
 \begin{itemize}
\item Usuario(id integer PRIMARY KEY, nombre varchar(30), sexo varchar(9), edad integer, dirección varchar(30), correo varchar(30), nacionalidad varchar(15))

\item TiendaProductos(id integer PRIMARY KEY, varchar(30), nombre varchar(30), ubicacion varchar(30), teléfono integer, correo varchar(30), rubro varchar(15)) 

\item TiendaServicios(id integer PRIMARY KEY, varchar(30), nombre varchar(30), ubicacion varchar(30), teléfono integer, correo varchar(30), rubro varchar(15), horario time)

\item Producto(id integer PRIMARY KEY, nombre varchar(30), precio integer, descripción varchar(30), tipo varchar(15))

\item Servicio(id integer PRIMARY KEY, nombre varchar(30), precio integer, descripción varchar(30), tipo varchar(15))

\item ProductoComprado(id integer PRIMARY KEY, id\_compra integer, id\_producto\_ integer, id\_tienda\_productos)

\item ServicioComprado(id integer PRIMARY KEY, id\_compra integer, id\_servicio integer, id\_tienda\_servicios, fecha\_limite date)

\item Compra(id integer PRIMARY KEY, fecha date, id\_usuario integer, review varchar(30))
\end{itemize}
 \section{Consultas}
 
	\subsection{}
%Se obtiene el id del comprador
$\rho$ $(Comprador,$ $\sigma_{nombre = nombre\_usuario}(\pi_{nombre}(Usuario))$


%Se obtiene tabla con los ids de las compras del comprador
$\rho$ $(Compras,$ $\pi_{id}(\sigma_{Comprador.id = id}(Compra))$

%Se obtiene los ids de las tiendas de las compras
$\rho$ $(TiendasProductos,$ $\pi_{id\_tienda\_productos}(Compras\bowtie_{compra.id = ProductoComprado.id\_compra}ProductoComprado)$

$\rho$ $(TiendasServicios,$ $\pi_{id\_tienda\_servicios}(Compras\bowtie_{Compras.id = ServicioComprado.id\_servicio}ProductoComprado)$

%Se obtienen los nombres de las tiendas
$\pi_{nombre} ((TiendasProductos \bowtie_{TiendasProductos.id\_tienda\_productos = TiendaProductos.id} TiendaProductos) \bigcup (TiendasServicios \bowtie_{TiendasServicios.id\_tienda\_servicios = TiendaServicios.id} TiendaServicios))$

\subsection{}

%Se obtienen todos los servicios comprados

$\rho$ $(TiendaComprada,$ $\sigma_{id_tienda_serivicios=id_tienda}(\pi_{id_tienda_servicios}(ServicioComprado)$

%obtener id de usuarios que han comprado en la tienda
$\rho$ $(UsuariosCompra,$ $\pi_{id_usuario}(TiendaComprada \bowtie_{TiendaComprada.id_compra = Compra.id}Compra))$

%filtrar por edad
$\sigma_{Usuario.edad >= 18}(UsuariosCompra\bowtie_{UsuariosCompra.id_usuario = Usuario.id}Usuario)$

\subsection{}

%encontrar todos los productos comprados dado el id

$\rho$ $(TodosProductos,$ $\sigma_{Producto.id=id_producto}(Producto \bowtie_{producto.id = ProductoComprado.id}ProductoComprado)$

%encontrar los usuarios a partir del id de compra

$\rho$ $(UsuariosCompra,$ $TodosProductos \bowtie_{TodosProductos.id_compra = Compra.id}Compra)$

%encontrar a los usuario y mostrar sus nombres

$\pi_{nombre}(UsuariosCompra \bowtie_{UsuariosCompra.id_usuario = Usuario.id}Usuario)$

\end{document}
